\documentclass{beamer}
\usetheme{CambridgeUS}
\usecolortheme{beaver}
\usepackage[utf8]{inputenc}
\usepackage{blindtext}
\usepackage{xcolor}
\usepackage{tikz}
\usepackage{verbatim}
\usepackage{hyperref}
\usepackage{textcomp}
\usepackage{framed}
\usepackage{datetime}

% create Date, Year
\newdateformat{monthyeardate}{%
  \monthname[\THEMONTH], \THEYEAR}

\title[Basic \LaTeX{}]{Using \LaTeX{} with Overleaf}
\author{Simone Lederer}
\institute[]
{Algorithms in Bioinformatics, ZBIT\\
  University of T\"{u}bingen \\ 
}
\date[Vorkurs WiSe 2020/21]{\monthyeardate\today, Prep-Bioinformatics Course}
\logo{\includegraphics[height=0.5cm]{figures/grey_unituebingen.png}\includegraphics[height=0.7cm]{figures/grey_MPI.png}}
\colorlet{palestgray}{gray!20}

\begin{document}
\setbeamertemplate{caption}{\raggedright\insertcaption\par}
\frame{\titlepage
\centering
\vspace{-10mm}
\url{https://github.com/Sim19/vorkurs_informatik_latex}}

\begin{frame} %[shrink=5]
\frametitle{Word processors and plain-tex}
\begin{columns}
\column[T]{0.48\textwidth}
    \begin{figure}
        % \centering
        \caption{A modern graphical word processor}
        \includegraphics[height=1.5cm]{figures/word-2016.png}
    \end{figure}
    \begin{figure}
        % \centering
        \caption{... and its best friend}
        \includegraphics[height=1.5cm]{figures/power_point_logo.jpg}
    \end{figure}
\column[T]{0.48\textwidth}
    \begin{figure}
        % \centering
        \caption{And the familiar problems...}
        \includegraphics[width=5.5cm, trim={0 0 0 0.5cm}]{figures/WordMemes.jpg}
    \end{figure}
\end{columns}
\end{frame}


\begin{frame}[shrink=15]
\frametitle{{\TeX} and {\LaTeX}}
    
\begin{columns}
    \column[T]{0.5\textwidth}
    \begin{itemize}
        \item \includegraphics[height=0.4cm]{figures/TeX_logo.png}
        \begin{itemize}
            \item designed by Donald E. Knuth in 1978
            \item a typographical system that does typesetting (tex core)
            \item set up stuff like how to place a graph or insert bibliographies (plain-tex)
        \end{itemize}
    \end{itemize}
    \begin{figure}
        \centering
            \includegraphics[height=4cm]{figures/Knuth.jpg}%
            \caption{Dr. Donald. E. Knuth}
    \end{figure}
    

    \column[T]{0.5\textwidth}
    \begin{itemize}
        \item \includegraphics[height=0.4cm]{figures/LaTeX_logo.png}
        \begin{itemize}
            \item written by Leslie Lamport, released in 1983
            \item a generalised set of macros built on top of {\TeX}
        \end{itemize}
    \end{itemize}
    \begin{figure}
        \centering
        \includegraphics[height=4cm]{figures/Lamport.jpg}
        \caption{Dr. Leslie Lamport}
    \end{figure}
\end{columns}
\end{frame}

%%%%%%%%%The bare minimum
\begin{frame}{The bare minimum}
The recipe:\\
\begin{enumerate}
    \item write source code in my favorite text editor
    \item execute source code: \texttt{pdflatex source\_code.tex}
\end{enumerate}
\pause
\centering
\color{palestgray}{\textit{\Huge{Things we won't do later on...}}}
\end{frame}


%%%%%introducing overleaf
\begin{frame}
\frametitle{Use Overleaf}
Steps: 1) click in the website and register
\hspace{6pt}%
2) start a new project
\begin{figure}
    \centering
    \includegraphics[width=10cm, trim=0 2.5cm 0 0, clip]{figures/Screen_shot_overleaf.png}
\end{figure}
\end{frame}

%%%%% The Bare Minimum
\begin{frame}[fragile]
\frametitle{\textit{Hello World} in Overleaf}
\begin{columns}
\column{0.45\linewidth}
\begin{framed}
\begin{verbatim}
\documentclass{article}
...
\begin{document}
...
\end{document}
\end{verbatim}
\end{framed}
\column{0.45\linewidth}

{\color{purple}\textbackslash}\hspace{5pt}
here goes the command\\
{\color{purple}\%}\hspace{5pt}
here comes the comment.\\
{\color{purple}\{\}}\hspace{5pt}
mandatory arguments;\\
tell \LaTeX{} this is a group\\\bigskip
{\small{\color{purple}\verb$\begin{...}$\\...\\\verb$\end{...}$}}\\
delimit an environment

\end{columns}
    
\end{frame}

%%%%%%hello world in overleaf (con'd)
\begin{frame}[fragile]
\frametitle{\textit{Hello World}{\small{(con'd)}}: \textit{document} environment}
\begin{columns}
\column{0.5\linewidth}
\textbf{\%Preamble}
\begin{verbatim}
\documentclass{article}
\usepackage[utf8]{inputenc}
\pagestyle{empty}
\end{verbatim}
\smallskip
{\color{purple}\verb$\begin{document}$}\\
\bigskip
\textbf{\%Top matter}
\begin{verbatim}
\title{Tardis User Manual}
\author{Dr. Who}
\date{October 2019}
\maketitle
\end{verbatim}
{\color{purple}\verb$\end{document}$}
\column{0.45\linewidth}
\includegraphics[width=0.95\columnwidth]{figures/example_extrm_simple_doc.pdf}
\end{columns}
   
\end{frame}


%%%%exercise 001
\begin{frame}
\frametitle{Exercise 01}
\centering
{\Huge{Simple.tex}}\par
\vspace{12pt}make the simplest document with title and author info
    
\end{frame}

%%%%Structure of an article
\begin{frame}
\frametitle{Structure of an article}
In principle, all elements of a document can be controlled with commands.\par
In an article:\par
\begin{itemize}
    \item Abstract
    \item Paragraphs. Maybe hierarchical
    \item Tables, figures
    \item Bibliography
    \item Page layout like headers, footers, margins
\end{itemize}
\end{frame}

\begin{frame}[fragile]
\frametitle{Structure of an article \small{(con{\textquotesingle}d)}}
\textbf{Abstract}
delimited as an environment
\begin{columns}
\column{0.5\linewidth}
\begin{verbatim}
\begin{abstract}
Insert text here. 
\end{abstract}
\end{verbatim}
\column{0.45\linewidth}
\includegraphics[width=0.95\columnwidth]{figures/abstract.png}
\end{columns}
\vspace{15pt}What can be changed:
\begin{itemize}
    \item title of abstract (default: Abstract)
    \item text alignment
    \item margin
\end{itemize}
\end{frame}

%%%%%Structure of an article: paragraph hierarchy
\begin{frame}[fragile]
\frametitle{Structure of an article \small{(con{\textquotesingle}d)}}
\textbf{About text conjuction}
\begin{itemize}
    \item \LaTeX{} ignores leading and trailing whitespaces/tabs
    \item line break is triggered by this command {\verb$\\$}, not by pressing ENTER
    \item new paragraph command {\verb$\par$}
    \item new line $\neq$ new paragraph
\end{itemize}
\textbf{Paragraph Sectioning}
\begin{itemize}
    \item  {\verb$\section{Title here}$} \par
    No need to markup the block like it in \textit{Abstract}.\par
    \item sections are numbered\par
    use {\verb$\section*{Title here}$} to un-number.
\end{itemize}
\end{frame}

\begin{frame}
\frametitle{Structure of an article \small{(con{\textquotesingle}d)}}
\textbf{Section hierarchy}\par
\vspace{12pt}
\begin{columns}
\column{0.5\linewidth}
\begin{tabular}{l|l}
   1  & section \\
   1.1  & subsection\\
   1.1.1 & subsubsection
\end{tabular}\par
\column{0.5\linewidth}
All these levels are above paragraph. 
Subparagraph is below paragraph (visible differences in indention).
\end{columns}
\vspace{12pt}Not all sectioning are implemented in a certain document type. e.g., 
most of the above are not in letters.
\end{frame}

%%%%exercise 02: creating an article
\begin{frame}[fragile]
\frametitle{Exercise 02}
\centering
{\Huge{Article.tex}}\par
\vspace{12pt}make an article with abstract and sectioning\par\vfill
Dummy text can be generated from package \texttt{blindtext} or \texttt{lipsum}\par
\hspace{12pt}{\verb$\lipsum[1-3]$} or {\verb$\blindtext$}
\end{frame}



%%%%% Paragraph formatting
\begin{frame}
\frametitle{Paragraph formatting}
\begin{columns}
    \column[T]{0.48\linewidth}
    \textbf{Capabilities}\vspace{12pt}
    \begin{itemize}
        \item text alignment
        \item indentation
        \item spacing between lines
        \item spacing between paragraphs
    \end{itemize}
    \column[T]{0.48\linewidth}
    \textbf{Specially...}\vspace{12pt}
    % \begin{columns}
        % \column{0.5\columnwidth}
        \includegraphics[width=0.33\columnwidth]{figures/special_parag_01.png}
        % \column{0.5\columnwidth}
        \includegraphics[width=0.33\columnwidth]{figures/special_parag_02.png}
        % \column{0.5\columnwidth}
        \includegraphics[width=0.33\columnwidth]{figures/special_parag_04.png}
    % \end{columns}
\vspace{12pt}
\includegraphics[width=0.98\columnwidth]{figures/special_math_formula.png}\vspace{12pt}
\end{columns}
    
\end{frame}
%%%%%paragraph formatting (con'd 01)
\begin{frame}[fragile]
\frametitle{Paragraph formatting \small{(con{\textquotesingle}d)}}
\textbf{text alignment}\par
\begin{itemize}
\small
    \item justified, ragged; to the left/right/center.
    \item Normally paragraphs are \emph{flushed} on both ends, i.e. left and right \emph{justified}
    \item change alignment using an environment or {\verb$\raggedright{}$} command
\end{itemize}\par
\footnotesize
  \begin{verbatim}
     \begin{flushleft}
     content here.            \raggedright{content here.}
     \end{flushleft}
\end{verbatim}

\begin{itemize}
    \item Centering: {\verb$\centering{content here.}$}, or
\end{itemize}
\begin{verbatim}
      \begin{center}
      content here.
      \end{center}
\end{verbatim}
    
\end{frame}
%%%%paragraph formatting (con'd 02)
\begin{frame}[fragile]
\frametitle{Paragraph formatting \small{(con{\textquotesingle}d)}}
\textbf{Spacing}\par
\begin{itemize}
    \item local changes: {\verb$\vspace{size}$}, {\verb$\smallskip \medskip \bigskip$}
    \item Aiming at a larger scope:\par
    put the commands inside an environment.\par
    It will take effect where its source code is written, and end with this environment
\end{itemize}
\footnotesize
\begin{verbatim}
    \setlength{\parskip}{6pt}   % space between paragraph = 6pt
    \renewcommand{\baselinestretch}{2}  % space between lines twice
                                        the current size
\end{verbatim}
\normalsize
\begin{itemize}
    \item global changes: override default in the preamble
\end{itemize}
\vfill
\end{frame}

\begin{frame}[fragile]
\frametitle{Paragraph formatting \small{(con{\textquotesingle}d)}}
\textbf{Indentation}
\begin{itemize}
    \item \LaTeX doesn't care how many whitespaces you typed at the beginning of a line.
    \item Implementation of indention varies among doc types and environments
    \item Override the default: in the same manner as \emph{spacing}
\end{itemize}
\vspace{12pt}
\hspace{2em}Add some horizontal space: {\verb$\hspace{length}$}\par
\hspace{2em}Change settings: {\verb$\setlength{\parindent}{length}$}
\vfill
\end{frame}

\begin{frame}[fragile]
\frametitle{Paragraph formatting {\small (con{\textquotesingle}d)}: Listing}
\begin{columns}
    \column{0.75\linewidth}
    \centering
    \includegraphics[width=0.2\columnwidth]{figures/special_parag_01.png}
    \includegraphics[width=0.2\columnwidth]{figures/special_parag_02.png}
    \includegraphics[width=0.2\columnwidth]{figures/special_parag_04.png}
\end{columns}
\bigskip
Types: \emph{enumerate}, \emph{description}, \emph{itemize}
\footnotesize
\begin{columns}
\column{0.3\linewidth}
\begin{verbatim}
\begin{enumerate}
\item item_1
\item item_2
\item item_3
\end{enumerate}
    \end{verbatim}
    \column{0.4\linewidth}
    \begin{verbatim}
\begin{description}
\item [item_1] about item_1
\item [item_2] about item_2
\item [item_3] about item_3
\end{description}
    \end{verbatim}
    \column{0.25\linewidth}
        \begin{verbatim}
\begin{itemize}
\item item_1
\item item_2
\item item_3
\end{itemize}
    \end{verbatim}
\end{columns}
\end{frame}


%%%%exercise 03: Format paragraphs
\begin{frame}
\frametitle{Exercise 03}
\centering
{\Huge{Paragraph.tex}}\par
\vspace{12pt}Format paragraphs. Change indentation, spacing, add some bullet points.
\end{frame}



\begin{frame}[fragile]
\frametitle{\LaTeX \hspace{1ex}measurement units}
\begin{columns}
    \column[T]{0.5\linewidth}
    \textbf{Absolute measurements}\\\vspace{6pt}
    \includegraphics[width=0.9\columnwidth]{figures/latex_lengths.png}
    \column[T]{0.5\linewidth}
    \textbf{Predefined lengths}\par
    \footnotesize
    \begin{verbatim}
\parskip    \parindent
\smallskip  \bigskip    \medskip
\textwidth  \linewidth  \columnwidth
    \end{verbatim}
    \begin{itemize}
        \item More flexible when fitting figures and tables
        \item relative measurements
    \end{itemize}
        
\end{columns}
    
\end{frame}



\begin{frame}
\frametitle{Text formatting}
\textbf{Capabilities}\par
\begin{center}
    {\LARGE font size}\hspace{18pt}\textrm{font styles}\\
    \vspace{15pt}{\color{cyan}{color}}\hspace{15pt}éñçødîng \\\vspace{15pt}
    \hspace{15pt} \$pe\textcopyright ial ch$\alpha$rac\dag ers
\end{center}
\vfill
\color{gray}{\footnotesize{Only the basics are discussed here. Packages are available for more advanced text tuning.}}
\end{frame}

\begin{frame}[fragile]
\frametitle{\emph{Scope}: an example}
Tell the difference:
\begin{verbatim}
    \centering                      {\centering
    content here.\par               content here.\par
    other stuff here.\par           other stuff here.\par}
    More and more.                  More and more.
\end{verbatim}
\vfill
\texttt{\{\}} can be used to delimit a group
\end{frame}


\begin{frame}[fragile]
\frametitle{Text formatting {\small (con{\textquotesingle}d)}: Font Style}
\textbf{Font families}
\begin{small}
\begin{itemize}
    \item {\verb$\textrm{content}$}\hspace{12pt} \textrm{Roman}
    \item {\verb$\textsf{content}$}\hspace{12pt} \textsf{Sans Sarif}
    \item {\verb$\texttt{content}$}\hspace{12pt} \texttt{Monospace}
\end{itemize}
\end{small}
\textbf{Shapes}
\begin{small}
\begin{itemize}
    \item {\verb$\textit{content}$}\hspace{12pt} \textit{Italic}
    \item {\verb$\textbf{content}$}\hspace{12pt} \textbf{Bold}
    \item {\verb$\textsl{content}$}\hspace{12pt} \textsl{Slanted} {\footnotesize(difference to italic is visible in \textrm{Roman} font)}
    \item {\verb$\emph{content}$}\hspace{12pt} \emph{Emphasized} 
\end{itemize}
\end{small}
Use packages like \textit{fontspec} for advanced font customization.
\vfill
Scope: just \emph{content}
\end{frame}

%%%%%text formatting: font size
\begin{frame}[fragile]
\frametitle{Text formatting {\small (con{\textquotesingle}d)}: Font Size}
In practice, 2 options for built-in font sizes:
\begin{itemize}
    \item Absolute sizes in {\verb$\documentclass[xpt]{article}$}\\
    {\small options: 10pt, 11pt, 12pt for \textit{article}, \textit{report}, \textit{book}}
    \item Built-in font size command
\end{itemize}
\small
\begin{verbatim}
    \tiny  \scriptsize  \footnotesize  \small
                   \normalsize
    \large  \Large  \LARGE  \huge  \Huge
\end{verbatim}
\normalsize
\textbf{Scope:} Till the end of its environment if not delimited\par
\vfill
For arbitrary font size, use {\verb$\fontsize{cur_font_size}{line_spacing_size}$}.
\end{frame}


%%%%%text formatting: color
\begin{frame}[fragile]
\frametitle{Text formatting {\small (con{\textquotesingle}d)}: color}
{\verb$\usepackage{xcolor}$}\par
\begin{itemize}
    \item preceded by package \texttt{color}, which is less flexible.
    \item basic command: {\verb$\color{what_color}{content_to_color}$}\\
    {\small {\verb$\textcolor{}{}$}: same except that it does not allow nesting environments}
\end{itemize}

\textbf{built-in colors}\hspace{12pt}
\textit{black  white  blue  red  gray  green  yellow...}\par\vspace{6pt}
\textbf{Scope} \hspace{12pt}To the end of current environment\par\vspace{6pt}
\textbf{Mix your own color}\par
\definecolor{prettyorange}{HTML}{FF7F00}
    \hspace{3em}{\verb$\definecolor{name}{model}{how_to_make_in_cur_model}$}\\
    \hspace{3em}{\color{prettyorange}{\verb$\definecolor{prettyorange}{HTML}{FF7F00}$}}
\vfill    
\end{frame}


%%%%%text formatting: special characters
\begin{frame}[fragile]
\frametitle{Text formatting {\small (con{\textquotesingle}d)}: special characters}
\begin{columns}
\column[T]{0.48\linewidth}
\textbf{Capabilities}
\begin{itemize}
    \item Characters to \textit{escape}\\
    \texttt{\%\hspace{8pt}\textbackslash\hspace{8pt}\{\}\hspace{8pt}\$\hspace{8pt}\textunderscore\hspace{8pt}\textgreater}
    \item The untypeable\\
    \texttt{\dag\hspace{8pt}\pounds\hspace{8pt}\texttrademark\hspace{8pt}\S\hspace{8pt}\textquestiondown}
    \item Math\\
    $\forall$\hspace{8pt}$\infty$\hspace{8pt}$\neq$\hspace{8pt}$\subset$\hspace{8pt}$\supset$
    \item Encoding-related\\
    \aa\hspace{8pt}\ss\hspace{8pt}$\ddot{u}$
\end{itemize}
\column[T]{0.5\linewidth}
\textbf{How-to}
\small
\begin{itemize}
    \item As command in text mode: {\verb$\%  \textbackslash  \pounds$}
    \item As command in math mode: {\verb#$\forall$  $\subset$  $\neq$#}
    \item Google for commands
    \item One symbol might have several implementations  in basic \LaTeX{} and extended packages
\end{itemize}
\end{columns}
\vfill
\end{frame}


\begin{frame}
\frametitle{Text formatting {\small (con{\textquotesingle}d)}: special characters - life hack}
visit \url{https://detexify.kirelabs.org/classify.html} and draw symbol
\begin{columns}
\column[T]{.28\linewidth}
\begin{figure}
\includegraphics[height = 4cm]{figures/detexify_start}
\end{figure}
\column[T]{.68\linewidth}
\begin{figure}
\includegraphics[height = 3.8cm]{figures/detexify_results}
\end{figure}
\end{columns}
\end{frame}

%%%%exercise 04: Text paragraphs
\begin{frame}
\frametitle{Exercise 04}
\centering
{\Huge{TextFormat.tex}}\par
\vspace{12pt}Format: \texttt{YOU}\hspace{1em}{\LARGE are}\hspace{1em}{\color{red}{B\emph{e}ing}}\hspace{1em}\underline{W}$\alpha$t\textcopyright hed
\end{frame}


%%%%%Bibliography
\begin{frame}
\frametitle{Bibliography}
\textbf{Structure of a citation}\par
1) reference information; 2) position in the main text.\par
Connected by reference number (\emph{key}).\par
\vspace{12pt}
\textbf{Information needed by \LaTeX{} in order to cite}\par
1) reference information; 2) position in the main text\par
Given a reference alias as \emph{key}.\par
Reference numbering is already implemented.\par
\vspace{12pt}
\textbf{Capabilities}\par
Bibliographic style
\vfill
\end{frame}


%%%%%Bibliography: two ways to do it
\begin{frame}[fragile]
\frametitle{Bibliography{\small (con{\textquotesingle}d)}}
\textbf{Reference information embedded}\par
\begin{itemize}
    \item \texttt{thebibliography} environment keeps reference information
    \item bibliographic entry: {\verb$\bibitem{thekey}$};\\main text: {\verb$\cite{thekey}$}
    \item manually set the format of references
\end{itemize}\vfill
\textbf{Import from external .bib files}\par\vspace{4pt}
Tool: \emph{BibTex}\par
Overleaf simplified the procedure to be:
    \begin{enumerate}
        \item Import: {\verb$\bibliography{file_path_no_need_extension}$}
        \item Set style: {\verb$\bibliographystyle{style_name}$}
        \item Cite
    \end{enumerate}
\end{frame}

%%%%%Bibliography: style
\begin{frame}[fragile]
\frametitle{Bibliography{\small (con{\textquotesingle}d)}: \emph{.bib} file}
They can be separated: {\verb$\bibliography{file1,file2,file3}$}\par\vspace{4pt}
Or one file containing information of several references
\footnotesize
\begin{verbatim}
@article{thekey,
  title={Full Title},
  author={Vorname1, Name1 and Vorname2, Name2 and Vorname3, Name3},
  journal={Journal Name},
  volume={number},
  number={number},
  pages={page_number},
  year={2019},
  publisher={Publisher Name}
}

@book{thekey,
   ...
   ...
   ...
}
\end{verbatim}
\end{frame}

\begin{frame}[fragile]
\frametitle{Bibliography{\small (con{\textquotesingle}d)}: Choose citation style}
Styles in \emph{BibTex}:\\
\texttt{unsrt}, \texttt{plain}, \texttt{abbrv}, \texttt{acm}, \texttt{alpha}, \texttt{apalike}\\
\vspace{12pt}
\textbf{Package \textit{natbib}}\\
\begin{itemize}
    \item modified {\verb$\cite{}$} to work with both author–year and numerical citations
    \item basic command {\verb$\citet{}$} and {\verb$\citep{}$}.
    \item add $\ast$ to list all authors in the main text.
\end{itemize}
\centering
\includegraphics[height=2.5cm]{figures/natbib_screenshot.png}
\end{frame}


%%%%exercise 05: Citation
\begin{frame}
\frametitle{Exercise 05}
\centering
{\Huge{Bibliography.tex}}\\\vspace{8pt}
Import citation via \emph{BibTex}; customize style with \texttt{natbib}
\end{frame}


%%%%Insert images
\begin{frame}[fragile]

\frametitle{Insert images}
{\verb$\usepackage{graphicx}$}\par
\begin{itemize}
    \item Package \texttt{graphics} extended
    \item inform \LaTeX{} where is the image file:\\ {\verb$\graphicspath{dir}$} (default: pwd)
    \item include it in the document:\\ {\verb$\includegraphics[size_param]{imagefile}$}
\end{itemize}

\end{frame}


%%%%Insert images: Problems
\begin{frame}[fragile]
\frametitle{Insert images {\small (con{\textquotesingle}d)}: Problems}
\begin{itemize}
\item Positioning among text 
    \item Positioning at a page break
    \item Caption, cross-reference...\vspace{8pt}
    \footnotesize
    \begin{columns}
    \column[T]{0.5\linewidth}
    A casual image:\\
    \centering
    \includegraphics[height=2.5cm]{figures/Emmentaler_aoc_block.jpg}
    \column[T]{0.5\linewidth}
    A scientific figure:
    \begin{figure}
        \centering
        \includegraphics[height=2.5cm]{figures/Emmentaler_aoc_block.jpg}
        \caption{\textrm{Figure 1: Spongebob Squarepants}}
    \end{figure}
    \end{columns}
\end{itemize}

\end{frame}


%%%% Insert images: Size and positioning
\begin{frame}[fragile]
\frametitle{Insert images {\small (con{\textquotesingle}d)}: Size and positioning}
\footnotesize
{\verb$\includegraphics[height=3cm, width=5cm, scale=1.2, angle=45]{pic.png}$}\par
\begin{itemize}
    \item Built-in length measurements are all acceptable
    \item Positions correspondent to source code
    \item No captions attached
\end{itemize}
\end{frame}

%%%% Insert images: figure environment
\begin{frame}[fragile]
\frametitle{Insert images {\small (con{\textquotesingle}d)}: \emph{figure} environment}
\scriptsize
\begin{verbatim}
    \begin{figure}[pos_options]
    % position options: h (here), b (bottom), t (top), p (put in a page)
    
        \centering
        \includegraphics[scale=0.3]{pic.png}
        \caption{Caption this}
        \label{fig:my_label} % a key for in-text referencing
        
    \end{figure}
\end{verbatim}
\normalsize
\textbf{Wrap text around figures:}
\scriptsize \verb$\usepackage{wrapfig}{alignment}{size}$
% \begin{verbatim}
%         \documentclass{article}
%         \usepackage{wrapfig}{alignment}{size}
%         ...
%         \begin{document}
%         \begin{wrapfigure}
%         ...
%         \end{wrapfigure}
%         ...
%         \end{document}
% \end{verbatim}
\end{frame}

%%%%%Floats
\begin{frame}[fragile]
\frametitle{\emph{Floats}}

\textbf{Definition:} anything within a document that cannot be broken over a page. Or roughly, tables and figures.\par\vspace{20pt}
\textbf{\LaTeX{}\textquoteright s solution}
\begin{itemize}
    \item If running out of space in current page, float the float to the next page
    \item Fill current page with body text
    \item {\verb$\begin{figure}[p]$} \hspace{3pt}  Gather figures to a float-only special page.
\end{itemize}
\vfill
\end{frame}


%%%%%Tables
\begin{frame}
\frametitle{Add tables}
\textbf{elements formatting a table}
\begin{itemize}
    \item Float-related properties like figures (positioning, size)
    \item Interior design: layout
    \begin{itemize}
        \item number of columns and rows
        \item line styles
        \item text adjusting
    \end{itemize}
\end{itemize}
\vfill   
\end{frame}


%%%%%tables(con'd): basic to advanced
\begin{frame}[fragile]
\frametitle{Add tables{\small (con{\textquotesingle}d)}: \emph{tabular} environment}
\begin{columns}
\column[T]{0.5\linewidth}
\scriptsize
\begin{verbatim}
  \begin{tabular}{ l || c | r | }
    \hline
    Table & Col_1 & Col_2 \\ \hline
    Row_1 & 34 & 41 \\ \hline
    Row_2 & 0.25 & 0.08 \\
    \hline
  \end{tabular}
\end{verbatim}
\normalsize
\begin{itemize}
    \item Column delimiter: \texttt{\&}
    \item Alignment: \texttt{l  c  r}
    \item Vertical line: \textbar\hspace{8pt} horizotal line: {\verb$\hline$}
\end{itemize}
\column[T]{0.5\linewidth}
\scriptsize
\begin{verbatim}
\begin{tabular*}{\textwidth}{ | l | r | }
\hline
col_1 & col_2 \\
\hline 
item_1  & item_2 \\
\hline
\end{tabular*}
\end{verbatim}
\normalsize
A subtle extension:

\begin{itemize}
    \item Specifying table width is allowed
    \item Adjusting column width to fit in the fixed table width is allowed
\end{itemize}
\end{columns}
\end{frame}


%%%%%tables(con'd): basic to advanced
\begin{frame}[fragile]
\frametitle{Add tables{\small (con{\textquotesingle}d)}: More customization}
\begin{description}
\item[\emph{table} environment] Same as \emph{figure} env. For better placement of the table.
\item[Package \emph{array}] For width adjustment.
\item[Package \emph{multirow}] To merge rows in some columns.
\item[Package \emph{longtable}] For tables across pages.
\end{description}
\end{frame}


%%%%exercise 04: Add figures and tables
\begin{frame}
\frametitle{Exercise 05}
\centering
\Huge{FloatingElements.tex}
\end{frame}


\begin{frame}
\frametitle{}
\vfill
\centering
\Huge{Thanks!}
\end{frame}

\end{document}

Protein Data Bank

The Protein Data Bank (PDB) is a database of 3D structure data for large biological molecules, such as proteins, DNA, and RNA. PDB is managed by an international organization called the Worldwide Protein Data Bank (wwPDB), which is composed of several local organizations, as. PDBe, PDBj, RCSB, and BMRB. They are responsible for keeping copies of PDB data available on the internet at no charge. The number of structure data available at PDB has increased each year, being obtained typically by X-ray crystallography, NMR spectroscopy, or cryo-electron microscopy.

Data format

The PDB format (.pdb) is the legacy textual file format used to store information of three-dimensional structures of macromolecules used by the Protein Data Bank. Due to restrictions in the format structure conception, the PDB format does not allow large structures containing more than 62 chains or 99999 atom records.

The PDBx/mmCIF (macromolecular Crystallographic Information File) is a standard text file format for representing crystallographic information. Since 2014, the PDB format was substituted as the standard PDB archive distribution by the PDBx/mmCIF file format (.cif). While PDB format contains a set of records identified by a keyword of up six characters, the PDBx/mmCIF format uses a structure based on key and value, where the key is a name that identifies some feature and the value is the variable information.

Other structural databases

In addition to the Protein Data Bank (PDB), there are several databases of protein structures and other macromolecules. Examples include:

    MMDB: Experimentally determined three-dimensional structures of biomolecules derived from Protein Data Bank (PDB).
    Nucleic acid Data Base (NDB): Experimentally determined information about nucleic acids (DNA, RNA).
    Structural Classification of Proteins (SCOP): Comprehensive description of the structural and evolutionary relationships between structurally known proteins.
    TOPOFIT-DB: Protein structural alignments based on the TOPOFIT method.
    Electron Density Server (EDS): Electron-density maps and statistics about the fit of crystal structures and their maps.
    CASP: Prediction Center Community-wide, worldwide experiment for protein structure prediction CASP.
    PISCES server for creating non-redundant lists of proteins: Generates PDB list by sequence identity and structural quality criteria.
    The Structural Biology Knowledgebase: Tools to aid in protein research design.
    ProtCID: The Protein Common Interface Database Database of similar protein-protein interfaces in crystal structures of homologous proteins.
